%
%  untitled
%
%  Created by Andra Waagmeester on 2013-09-24.
%  Copyright (c) 2013 Maastricht University. All rights reserved.
%
\documentclass[]{article}

% Use utf-8 encoding for foreign characters
\usepackage[utf8]{inputenc}

% Setup for fullpage use
\usepackage{fullpage}

% Uncomment some of the following if you use the features
%
% Running Headers and footers
%\usepackage{fancyhdr}

% Multipart figures
%\usepackage{subfigure}

% More symbols
%\usepackage{amsmath}
%\usepackage{amssymb}
%\usepackage{latexsym}

% Surround parts of graphics with box
\usepackage{boxedminipage}

% Package for including code in the document
\usepackage{listings}

% If you want to generate a toc for each chapter (use with book)
\usepackage{minitoc}

% This is now the recommended way for checking for PDFLaTeX:
\usepackage{ifpdf}

%\newif\ifpdf
%\ifx\pdfoutput\undefined
%\pdffalse % we are not running PDFLaTeX
%\else
%\pdfoutput=1 % we are running PDFLaTeX
%\pdftrue
%\fi

\ifpdf
\usepackage[pdftex]{graphicx}
\else
\usepackage{graphicx}
\fi
\usepackage{tikz}
\usetikzlibrary{arrows}
\usepackage{listings,etoolbox}
\usepackage{fancyvrb}
\usepackage{hyperref}
\usepackage{lipsum}
% \usepackage[many]{tcolorbox}
\usepackage{color}
\usepackage{everypage}
\usepackage{draftwatermark}
\usepackage{biblatex}


\title{WikiPathways RDF Interactions}
\author{ Andra Waagmeester }

\date{2013-09-24}

\begin{document}

\ifpdf
\DeclareGraphicsExtensions{.pdf, .jpg, .tif}
\else
\DeclareGraphicsExtensions{.eps, .jpg}
\fi



\makeatletter
\lst@Key{lastline}\relax{\ifnumcomp{#1}{<}{0}{%
  \let\mylst@file\lst@intname\sbox0{\lstinputlisting{\mylst@file}}%
  \def\lst@lastline{\the\numexpr#1+\value{lstnumber}-1\relax}}%
  {\def\lst@lastline{#1\relax}}}
\makeatother
\maketitle

\SetWatermarkScale{4}
% Language Definitions for Turtle
\definecolor{olivegreen}{rgb}{0.2,0.8,0.5}
\definecolor{grey}{rgb}{0.5,0.5,0.5}
\lstdefinelanguage{ttl}{
sensitive=true,
morecomment=[l][\color{grey}]{@},
morecomment=[l][\color{olivegreen}]{\#},
morestring=[b][\color{blue}]\",
}

% Language Definition for GPML

\definecolor{gray}{rgb}{0.4,0.4,0.4}
\definecolor{darkblue}{rgb}{0.0,0.0,0.6}
\definecolor{cyan}{rgb}{0.0,0.6,0.6}

\lstset{
  basicstyle=\ttfamily,
  columns=fullflexible,
  showstringspaces=false,
  commentstyle=\color{gray}\upshape
}

\lstdefinelanguage{GPML}
{
  morestring=[b][\color{red}]",
  morestring=[s]{>}{<},
  morecomment=[s]{<?}{?>},
  % morecomment=[n][\color{red}]{"}{"}},
  stringstyle=\color{black},
  identifierstyle=\color{darkblue},
  keywordstyle=\color{cyan},
  morekeywords={xmlns,version,type,TextLabel,GraphId,Type,Point, X, Y, RelX, RelY, ArrowHead,
        Database, Width, Height, ZOrder, GraphRef, LineThickness,
        CenterX, CenterY, FontSize, Valign}% list your attributes here
}

% Language Definitions for SPARQL
\lstdefinelanguage{sparql}{
morecomment=[l][\color{olivegreen}]\#,
morestring=[b][\color{blue}]\",
morestring=[s][\color{red}]{?}{\ },
morekeywords={SELECT, DISTINCT ,CONSTRUCT,DESCRIBE,ASK,WHERE,FROM,NAMED,PREFIX,BASE,OPTIONAL,FILTER,GRAPH,LIMIT,OFFSET,SERVICE,UNION,EXISTS,NOT,BINDINGS,MINUS,a},
sensitive=true
}
\tableofcontents

\newcommand{\docchapter}[1] {
\subsection{#1}
\subsubsection{Pathway snippet}
\includegraphics{#1.png}
\subsubsection{Description}
\input{#1.txt}
\subsubsection{GPML}
% Language Definition additions for XML
\lstset{
  language=GPML,
}
\lstinputlisting[caption=#1, firstline=4,lastline=-4,breaklines=true, frame=single]{#1.gpml}
\subsubsection{RDFSchema}
\includegraphics[width=\linewidth]{#1.pdf}
\subsubsection{RDF snippet (turtle)}
\lstset{
  language=ttl,
}
\lstinputlisting[caption=#1,breaklines=true, frame=single]{#1.ttl}

\subsubsection{Example SPARQL query}
\lstset{
  language=sparql,
}
\lstinputlisting[caption=#1,breaklines=true, frame=single]{#1.sparql}
\newpage
}



% Basic Interaction 

\section{External links}
\begin{itemize}
	\item pathway-template.svg \url{https://github.com/wikipathways/pathvisio.js/blob/master/src/views/pathway-template.svg}
	\item Formal mim specs: \url{http://discover.nci.nih.gov/mim/formal_mim_spec.pdf}
\end{itemize}

\section{Interactions}
In \textsc{WikiPathways/Gpml} there are basically two types of interaction templates. The first being two DataNodes connected by a graphical line.

\begin{tikzpicture}
	\node (datanode1) at (1,0)  [draw,thick,minimum width=80,minimum height=20] {};
	\node (datanode2)  at (6,0) [draw,thick,minimum width=80,minimum height=20] {} edge (datanode1);
	\draw [--] (datanode1) -- (datanode2);
\end{tikzpicture}
\\
The second being at least three DataNodes interacting with each other, where at least one interaction is interacting with another interaction.  
\\
\begin{tikzpicture}
	\node (datanode1) at (1,3)  [draw,thick,minimum width=80,minimum height=20] {};
	\node (datanode2)  at (6,0) [draw,thick,minimum width=80,minimum height=20] {};
	\node (edge1) at at (6,3) {} ;
	\node (datanode3)  at (6,6) [draw,thick,minimum width=80,minimum height=20] {};
	\draw [--]  (datanode2) -- (datanode3);
	\draw [--] (datanode1) -- (edge1);
\end{tikzpicture} 
 
There are 21 different interaction types being recognized. These are Line, Arrow, TBar, Receptor, LigandSquare, ReceptorBound, MimNecessaryStimulation, MimBinding, MimConverstion, MimStimulation,
MimModification, MimCatalysis, MimInhibition, MimCleavage, MimiCovalentBond, MimBranchingLeft, MimBranchingRight, MimTranscriptionTranslation, MimGap.

So called markers on the lines indicating an interaction gives both directionality and the type of interaction. 

\docchapter{Line}
\docchapter{Arrow}
\docchapter{TBar}
\docchapter{Receptor}
\docchapter{LigandSquare}
\docchapter{ReceptorSquare}
\docchapter{LigandRound}
\docchapter{ReceptorBound}
\docchapter{MimNecessaryStimulation}
\docchapter{MimBinding}
\docchapter{MimConversion}
\docchapter{MimStimulation}
\docchapter{MimModification}
\docchapter{MimCatalysis}
\docchapter{MimInhibition}
\docchapter{MimCleavage}
\docchapter{MimCovalentBond}
\docchapter{MimBranchingLeft}
\docchapter{MimBranchingRight}
\docchapter{MimTranscriptionTranslation}
\docchapter{MimGap}





\bibliographystyle{plain}
\bibliography{}
\end{document}
